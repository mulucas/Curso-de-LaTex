\documentclass{article}

\usepackage[utf8]{inputenc} % Este pacote serve para acentuação
\usepackage[brazil]{babel} % Este pacote coloca os nomes em pt-br
\usepackage{indentfirst} % Este pacote aplica indentação
\usepackage[a4paper, left=1cm, right=2cm, top=2cm, bottom=3cm]{geometry} % Este pacote altera a margem do documento
\usepackage{graphicx} % Este pacote permite adicionar figuras
\usepackage{float} % Força o posicionamento
\usepackage{multirow} % Mesclar linhas
\usepackage{tabularx} % Margem da tabela
\usepackage{amsmath} % Modo matemático

\begin{document}	
	\title{\textbf{{\Huge Modo matemático - erros comuns}}} % Título
	\author{Alexandre Nunes} % Autor
	\date{} % Data
	\maketitle % Criar o título, autor e data
	\thispagestyle{empty} % Oculta a numeração da página
	\newpage
	
	\setcounter{page}{1} % Começa a contar as páginas novamente
	\pagenumbering{Roman} % altera para algarismo romano
	\tableofcontents % Cria o sumário
	\newpage
	
	\listoffigures % Lista de figuras
	\newpage
	
	\listoftables % Lista de tabelas
	\newpage

	\setcounter{page}{1} % Começa a contar as páginas novamente
	\pagenumbering{arabic} % altera para algarismo arábico
	
	\section{Modo matemático}

    Esta é a equação de segundo grau: $ ax^2 +bx +c=0 $. A solução é:
    \begin{equation*}
    	x = \frac{-b \pm \sqrt{b^2 -4ac}}{2a}
    \end{equation*}   

	
\end{document}