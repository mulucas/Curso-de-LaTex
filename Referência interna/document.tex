\documentclass{article}

\usepackage[utf8]{inputenc} % Este pacote serve para acentuação
\usepackage[brazil]{babel} % Este pacote coloca os nomes em pt-br
\usepackage{indentfirst} % Este pacote aplica indentação
\usepackage[a4paper, left=1cm, right=2cm, top=2cm, bottom=3cm]{geometry} % Este pacote altera a margem do documento
\usepackage{graphicx} % Este pacote permite adicionar figuras
\usepackage{float} % Força o posicionamento
\usepackage{multirow} % Mesclar linhas
\usepackage{tabularx} % Margem da tabela

\begin{document}	
	\title{\textbf{{\Huge Citações internas}}} % Título
	\author{Alexandre Nunes} % Autor
	\date{} % Data
	\maketitle % Criar o título, autor e data
	\thispagestyle{empty} % Oculta a numeração da página
	\newpage
	
	\setcounter{page}{1} % Começa a contar as páginas novamente
	\pagenumbering{Roman} % altera para algarismo romano
	\tableofcontents % Cria o sumário
	\newpage
	
	\listoffigures % Lista de figuras
	\newpage
	
	\listoftables % Lista de tabelas
	\newpage

	\setcounter{page}{1} % Começa a contar as páginas novamente
	\pagenumbering{arabic} % altera para algarismo arábico
	
	
	\section{Seção AAA}
	\label{sec:secAAA}
	A tabela \ref{tab:tab_sem_borda} não possui bordas externas. Esta é a seção \ref{sec:secAAA}.

	
	\begin{table}[H]
		\centering
		\caption[Legenda Curta]{Esta Legenda é longa}
		\label{tab:tab_sem_borda}
		\begin{tabular}{l|l|l}
			{\LARGE \textbf{A1}} & \textbf{B1} & \textbf{C1} \\ \hline
			\textbf{A2}          &             &             \\ \hline
			A3          &             &             \\
		\end{tabular}
	\end{table}

	\renewcommand{\arraystretch}{1.5}
	\begin{table}[H]
		\centering
		\caption[Legenda Curta]{Esta Legenda é longa}
		\label{tab:tab_com_borda}
		\begin{tabular}{|l|l|l|}
			\hline
			\textbf{A1} & \multicolumn{2}{l|}{\textbf{B1}} \\ \hline
			A2          &                 &                \\ \hline
			A3          &                 &                \\ \hline
		\end{tabular}
	\end{table}
	
	
	\begin{figure}[H]
		\centering
		\includegraphics[width=0.5\linewidth]{Figuras/fig1}
		\caption[Legenda curta]{Esta é uma legenda longa}
		\label{fig:fig1}
	\end{figure}
	
	\section{Seção BBB}	
	
	\begin{figure}[H]
		\centering
		\includegraphics[width=0.5\linewidth]{Figuras/fig3}
		\caption[Legenda curta]{Esta é uma legenda longa}
		\label{fig:fig3}
	\end{figure}
	
	Referência a equação (\ref{eq:emc}).
	
	\begin{equation}
	\label{eq:emc}
		e=mc^2
	\end{equation}
	
	
\end{document}